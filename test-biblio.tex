\documentclass[11pt]{article}
\usepackage[margin=1in]{geometry}
\usepackage{amsmath}
\usepackage[backend=biber,style=authoryear-comp,maxcitenames=2]{biblatex}
\usepackage{csquotes}

% Add a bibliography file
\addbibresource{refs.bib}

\title{Bibliography Test Document}
\author{Test Author}
\date{\today}

\begin{document}

\maketitle

\section{Introduction}

This document tests bibliography functionality with texMini. We cite some famous works here to ensure the bibliography system works correctly.

LaTeX was developed by \textcite{lamport1994latex}, building on \TeX{} by \textcite{knuth1984texbook}. The modern bibliography system uses biblatex and biber \parencite{lehman2022biblatex}.

\section{Mathematics}

The Pythagorean theorem states that $a^2 + b^2 = c^2$ for right triangles. This fundamental result appears in many mathematical contexts \parencite{euclid300elements}.

\section{Modern Typography}

Digital typography has evolved significantly since the early days of computer typesetting \parencite{knuth1979tex}. Modern systems provide sophisticated control over layout and typography.

\section{Citation Examples}

Here are various citation styles:

\begin{itemize}
    \item Textcite: \textcite{lamport1994latex} developed LaTeX.
    \item Parencite: LaTeX is a document preparation system \parencite{lamport1994latex}.
    \item Multiple citations: These works \parencite{knuth1984texbook,lamport1994latex,lehman2022biblatex} are foundational.
    \item Cite: According to \cite{knuth1984texbook}, \TeX{} is a typesetting system.
\end{itemize}

\section{Conclusion}

If this document compiles successfully with a properly formatted bibliography, then texMini's bibliography support is working correctly. The bibliography requires multiple compilation passes:

\begin{enumerate}
    \item First LaTeX run - processes citations, creates .aux file
    \item Biber run - processes bibliography database, creates .bbl file  
    \item Second LaTeX run - incorporates bibliography into document
    \item Third LaTeX run - resolves cross-references
\end{enumerate}

Each step creates intermediate files that must be preserved until the entire process completes.

\printbibliography

\end{document}
